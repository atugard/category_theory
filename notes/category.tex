\documentclass[11pt]{report}
\usepackage[utf8]{inputenc}
\usepackage{mathtools}
\usepackage{tikz-cd}
\usepackage{bbm}
\newcommand{\mcC}{\mathcal{C}}
\newcommand{\mcD}{\mathcal{D}}
\newcommand{\mor}{\text{mor}}
\newcommand{\ob}{\text{ob}}
\newtheorem{thm}{Theorem}[chapter]
\newtheorem{defn}[thm]{Definition} %arguments encased in square brackets define counters. Here thm is counted according to chapter, and defn is counted according to theorem.


\begin{document}
\chapter{Categories}
\begin{defn}
  A category $\mcC$ is given by the following information $(\ob(\mcC), \hom(\mcC), \circ)$, where
  \begin{itemize}
  \item (Objects) $\ob(\mcC)$ denotes the class of objects. \\
  \item (Morphisms) $\hom(\mcC)$ denotes the class of morphisms, of the form $f:A \to B$, where $A, B \in ob(\mcC)$. \\
  \item (Composition) $\circ$ denotes the binary operation of composition, which satisfies:
    \begin{itemize}
    \item (Closed) For any $f: A \to B, g:B \to C \in \hom(C)$, $g \circ f: A \to C$. \\
    \item (Associativity) Let $f,g$ be as above, and $h:C->D$ be another morphism, then $h \circ (g \circ f) = (h \circ g) \circ f$. \\
    \item (Identity) For every object $X \in \ob(\mcC)$ there exists an identity morphism $\mathbbm{1}_X:X \to X$ satisfying
      $\mathbbm{1}_B \circ f = f \circ \mathbbm{1}_A$.
    \end{itemize}
  \end{itemize}
\end{defn}

\begin{defn}
  Let $\mcC$ and $\mcD$ be categories. A covariant functor $F: \mcC \to \mcD$ is a mapping satisfying
  \begin{enumerate}
  \item For all objects $A \in \ob(\mcC)$ there exists a unique object $FA \in \ob(\mcD)$.
  \item For all morphisms $f:A \to B$ of $\mcC$ a unique morphism $Ff:FA \to FB$ of $\mcD$, such that:
    \begin{enumerate}
    \item $F\mathbbm{1}_A = \mathbbm{1}_{FA}$, for all objects $A$ of $\mcC$. \\
    \item $F(g \circ f) = Fg \circ Ff$, for all morphisms $g: B \to C$ of $\mcC$.
    \end{enumerate}
  \end{enumerate}
\end{defn}
\begin{defn}
  A contravariant functor $F: \mcC \to \mcD$ satisfies the definition of a covariant functor, except change 2. in Definition 1.2 to
  \begin{itemize}
    \item For all morphisms $f:A \to B$ of $\mcC$ a unique morphism $Ff:FB \to FA$ of $\mcD$,
  \end{itemize}
  and change 2.b to
  \begin{itemize}
    \item $F(g \circ f) = Ff \circ Fg$, for all morphisms $g: B \to C$ of $\mcC$.
  \end{itemize}
\end{defn}

\begin{defn}
  An endofunctor is a functor that maps a category to itself. 
\end{defn}

\begin{defn}
  Let $F: \mcC \to \mcD$ and $G: \mcD \to \mcC$ be functors, then a natural transformation $\eta: F \to G$ is a collection of morphisms satisfying
  \begin{itemize}
    \item For all objects $A$ of $\mcC$ we get a morphism $\eta_A: FA \to GA$ of objects in $\mcD$, and we call $\eta_A$ the component of $\eta$ at $A$. \\
    \item Components satisfy $\eta_B \circ Ff = Gf \circ \eta_A$, for all morphisms $f: A \to B$ in $\mcC$,
      i.e. the following diagram commutes:
      \begin{center}
        \begin{tikzcd}
          FA \arrow[d, "{Ff}"] \arrow[r, "{\eta_A}"] & GA \arrow[d, "{Gf}"] \\
          FB \arrow[r, "{\eta_B}"] & GB
        \end{tikzcd}
      \end{center}

  \end{itemize}
\end{defn}

\begin{defn}
Let $F: \mcC \to \mcD$ and $G: \mcD \to \mcC$ be functors, then a natural transformation $\eta: F \to G$ is a natural isomorphism if all of the components $\eta_A:FA \to GA$ are isomorphisms, for $A$ an object of $\mcC$.
\end{defn}


\begin{defn}
  Two functors $F: \mcC \to \mcD, G: \mcD \to \mcC$ are adjoint if for any object $A$ of category $\mcC$ and $B$ of category $\mcD$,
  $\hom_{\mcC}(A,G-)$ is naturally isomorphic to $\hom_{\mcD}(FA,-)$ and $\hom_{\mcC}(-,GB)$ is naturally isomorphic to  $\hom_{\mcD}(F-,B)$.
  That is for any morphisms $f:C \to D$ of category $\mcC$ and $g: \widetilde{D} \to \widetilde{C}$ of category $\mcD$ there exist natural transformations $\eta$, $\mu$ such that the following diagrams
  \begin{center}
  \begin{tikzcd}
    \hom_{\mcC}(C, GB) \arrow[d, "{\hom_{\mcC}(-, GB)}"] \arrow[r, "{\eta_C}"]
    & \hom_{\mcD}(FC, B) \arrow[d, "{\hom_{\mcC}(F-, B)}"] \\
    \hom_{\mcC}(D, GB) \arrow[r, "{\eta_D}"]
    & \hom_{\mcD}(FD, B) 
  \end{tikzcd}
  \end{center}
  \begin{center}
\begin{tikzcd}
    \hom_{\mcC}(A, G\widetilde{D}) \arrow[d, "{\hom_{\mcC}(A, G-)}"] \arrow[r, "{\mu_{\widetilde{D}}}"]
    & \hom_{\mcD}(FA, \widetilde{D}) \arrow[d, "{\hom_{\mcC}(FA, -)}"] \\
    \hom_{\mcC}(A, G\widetilde{C}) \arrow[r, "{\mu_{\widetilde{C}}}"]
    & \hom_{\mcD}(FA, \widetilde{C}) 
  \end{tikzcd}
\end{center}
  commute, and each of the components of $\eta$ and $\mu$ are isomorphisms, i.e. $\eta_C$, $\eta_D$, $\mu_{\widetilde{D}}$, $\mu_{\widetilde{C}}$ are isomorphisms.
\end{defn}
\begin{defn}
  A coherence condition is a collection of conditions requiring that various compositions of elementary morphisms are equal. 
\end{defn}
\begin{defn}
  Let $\mcC$ be a category. A monad on $\mcC$ is given by an endofunctor $T: \mcC \to \mcC$ together with natural transformations $\eta: \mathbbm{1}_\mcC \to T$ (where $\mathbbm{1}_\mcC$ denotes the identity functor on $\mcC$ and
  $\mu: T^2 \to T$ (where $T^2$ is the functor $T \circ T$ from $\mcC$ to $\mcC$). These are required to fulfill the following coherence conditions:
  \begin{itemize}
  \item $\mu \circ T \mu = \mu \circ \mu T$ (as natural transformations $T^3 \to T$), i.e. for all objects $A$ of $\mcC$ the following diagram commutes,
    \begin{center}
      \begin{tikzcd}
        T(T(T(A))) \arrow[d, "{\mu_{T(A)}}"] \arrow[r,"{T(\mu_A)}"] & T(T(A)) \arrow[d, "{\mu_A}"] \\
        T(T(A)) \arrow[r, "{\mu_A}"] & T(A)
      \end{tikzcd}
    \end{center} 
  \item $\mu \circ T \eta = \mu \circ \eta T = \mathbbm{1}_T $ (as natural transformations  $T \to T$, where $\mathbbm{1}_T$ denotes the identity transformation from $T$ to $T$), i.e.
    the following diagram commutes, for all objects $A$ of $\mcC$:
    \begin{center}
      \begin{tikzcd}
        T(A) \arrow[d, "{T(\eta_A)}"] \arrow[r,"{\eta_{T(A)}}"] & T(T(A)) \arrow[d, "{\mu_A}"] \\
        T(T(A)) \arrow[r, "{\mu_A}"] & T(A)
      \end{tikzcd}
    \end{center} 

  \end{itemize}
\end{defn}

\end{document}
