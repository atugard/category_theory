\documentclass[11pt]{report}
\usepackage[utf8]{inputenc}
\usepackage{mathtools}
\usepackage{tikz-cd}
\usepackage{bbm}
\newcommand{\mcC}{\mathcal{C}}
\newcommand{\mcD}{\mathcal{D}}
\newcommand{\mor}{\text{mor}}
\newcommand{\ob}{\text{ob}}
\newtheorem{thm}{Theorem}[chapter]
\newtheorem{defn}[thm]{Definition} %arguments encased in square brackets define counters. Here thm is counted according to chapter, and defn is counted according to theorem.

%todo: write entries for products and coproducts, limits and coliminits, yonedas lemma, monoidal category, monoid, monad, Exponential object, Closed cartesian category, 

\begin{document}
\chapter{Categories}
\begin{defn}
  A \textbf{category} $\mcC$ is given by the following information $(\ob(\mcC), \hom(\mcC), \circ)$, where
  \begin{itemize}
  \item (Objects) $\ob(\mcC)$ denotes the class of \textbf{objects}. \\
  \item (Morphisms) $\hom(\mcC)$ denotes the class of \textbf{morphisms}, of the form $f:A \to B$, where $A, B \in ob(\mcC)$. \\
  \item (Composition) $\circ$ denotes the binary operation of \textbf{composition}, which satisfies:
    \begin{itemize}
    \item (Closed) For any $f: A \to B, g:B \to C \in \hom(C)$, $g \circ f: A \to C$. \\
    \item (Associativity) Let $f,g$ be as above, and $h:C \to D$ be another morphism, then $h \circ (g \circ f) = (h \circ g) \circ f$. \\
    \item (Identity) For every object $X \in \ob(\mcC)$ there exists an identity morphism $\mathbbm{1}_X:X \to X$ satisfying
      $\mathbbm{1}_B \circ f = f \circ \mathbbm{1}_A$.
    \end{itemize}
  \end{itemize}
\end{defn}

\begin{defn}
  Let $\mcC$ be a category. The \textbf{dual} category, or \textbf{opposite} category $\mcC^{op}$ is given by:
  \begin{itemize}
  \item $\ob(\mcC) = \ob(\mcC^{op})$.
  \item $\hom_{\mcC^{op}}(X, Y) = \hom_{\mcC}(Y, X)$, for every $X, Y \in \ob(\mcC)$.
  \item $g \circ_{op} f = (f^{op} \circ g^{op})^{op}$, for $f: A \to B$, $g: B \to C$ in $\mcC^{op}$.
  \end{itemize}
  \begin{center}
    .
  \end{center}

\end{defn}

\begin{defn}
  An \textbf{initial object} of a category $\mcC$ is an object $I \in \ob(\mcC)$ such that for every object $X \in \ob(\mcC)$ there exists exactly one morphism $I \to X$.
\end{defn}

\begin{defn}
  A \textbf{terminal object} of a category $\mcC$ is an object $J \in \ob(\mcC)$ such that for every object $X \in \ob(\mcC)$ there exists precisely one morphism $X \to J$.
  It is the dual notion of an initial object. That is, if $I$ is an initial object in $\mcC$ then $J$ is a terminal object in $\mcC^{op}$.
\end{defn}

\begin{defn}
  Let $\mcC$ and $\mcD$ be categories. A \textbf{covariant functor } $F: \mcC \to \mcD$ is a mapping satisfying
  \begin{enumerate}
  \item For all objects $A \in \ob(\mcC)$ there exists a unique object $FA \in \ob(\mcD)$.
  \item For all morphisms $f:A \to B$ of $\mcC$ a unique morphism $Ff:FA \to FB$ of $\mcD$, such that:
    \begin{enumerate}
    \item $F\mathbbm{1}_A = \mathbbm{1}_{FA}$, for all objects $A$ of $\mcC$. \\
    \item $F(g \circ f) = Fg \circ Ff$, for all morphisms $g: B \to C$ of $\mcC$.
    \end{enumerate}
  \end{enumerate}
\end{defn}
\begin{defn}
  A \textbf{contravariant functor} $F: \mcC \to \mcD$ satisfies the definition of a covariant functor, except change 2. in Definition 1.2 to
  \begin{itemize}
    \item For all morphisms $f:A \to B$ of $\mcC$ a unique morphism $Ff:FB \to FA$ of $\mcD$,
  \end{itemize}
  and change 2.b to
  \begin{itemize}
    \item $F(g \circ f) = Ff \circ Fg$, for all morphisms $g: B \to C$ of $\mcC$.
  \end{itemize}
\end{defn}

\begin{defn}
  An \textbf{endofunctor} is a functor that maps a category to itself. 
\end{defn}


\begin{defn}
  Let $F: \mcC \to \mcD$ be a functor. Let $X \in \ob(\mcD)$ and $A \in \ob(\mcC).$ A \textbf{universal morphism} from $X$ to $F$ is a unique pair
  $(A, u: X \to FA)$ in $\mcD$ which satisfies the following property, called a \textbf{universal property}.
  For any morphism of the form $f: X \to FA'$ in $\mcD$, there exists a unique morphism $h: A \to A'$ such that the following diagram commutes:
  \begin{center}
    \begin{tikzcd}
      X \arrow[r, "{u}"] \arrow[rd, swap, "{f}"] & FA \arrow[d, "{Fh}"] \\
      & FA' 
    \end{tikzcd}
  \end{center}
\end{defn}

\begin{defn}
  Let $\mcC$ be a category with some objects $X$ and $Y$. A \textbf{product} of $X$ and $Y$ is an object $X \times Y$ together with a pair of morphisms
  $\pi_X: X \times Y\to X$, $\pi_Y: X \times Y\to Y$ that satisfy the following universal property:
  \begin{itemize}
  \item For every object $Z$ and a pair of morphisms $f: Z \to X$, $g: Z \to Y$ there exists a unique morphism $h: Z \to X \times Y$ such that
    the following diagram commutes:
    \begin{center}
      \begin{tikzcd}
        & Z \arrow[dl, swap, "{f}"] \arrow[d, "{h}"] \arrow[dr, "{g}"] & \\
        X & X \times Y \arrow[l,  "{\pi_X}"]  \arrow[r, swap,  "{\pi_Y}"] & Y
      \end{tikzcd}
    \end{center}
  \end{itemize}
\end{defn}

\begin{defn}
  Let $\mcC$ be a category with some objects $X$ and $Y$. A \textbf{coproduct} or \textbf{sum} of $X$ and $Y$ is an object $X \sqcup Y$ or $X \oplus Y$ or $X + Y$
  together with morphisms $i _X: X \to X \times Y$, $i_Y: Y \to X \times Y$ that satisfy the following universal property:
  \begin{itemize}
    \item For any object $Z$ and morphisms $f: X \to Z$, $g: Y \to Z$, there exists a unique morphism $h: X \sqcup Y \to Z$ such that the following diagram commutes.
    \begin{center}
      \begin{tikzcd}
        & Z & \\
        X \arrow[ur, "{f}"] \arrow[r, swap, "{i_1}"] & X \sqcup Y \arrow[u, swap, "{h}"] & Y \arrow[ul, swap, "{g}"] \arrow[l, "{i_2}"]
      \end{tikzcd}
    \end{center}
  \end{itemize}
\end{defn}


\begin{defn}
  Let $F: \mcC \to \mcD$ and $G: \mcD \to \mcC$ be functors, then a \textbf{natural transformation} $\eta: F \to G$ is a collection of morphisms satisfying
  \begin{itemize}
    \item For all objects $A$ of $\mcC$ we get a morphism $\eta_A: FA \to GA$ of objects in $\mcD$, and we call $\eta_A$ the \textbf{component }of $\eta$ at $A$. \\
    \item Components satisfy $\eta_B \circ Ff = Gf \circ \eta_A$, for all morphisms $f: A \to B$ in $\mcC$,
      i.e. the following diagram commutes:
      \begin{center}
        \begin{tikzcd}
          FA \arrow[d, "{Ff}"] \arrow[r, "{\eta_A}"] & GA \arrow[d, "{Gf}"] \\
          FB \arrow[r, "{\eta_B}"] & GB
        \end{tikzcd}
      \end{center}

  \end{itemize}
\end{defn}

\begin{defn}
Let $F: \mcC \to \mcD$ and $G: \mcD \to \mcC$ be functors, then a natural transformation $\eta: F \to G$ is a \textbf{natural isomorphism} if all of the components $\eta_A:FA \to GA$ are isomorphisms, for $A$ an object of $\mcC$.
\end{defn}


\begin{defn}
  Two functors $F: \mcC \to \mcD, G: \mcD \to \mcC$ are \textbf{adjoint }if for any object $A$ of category $\mcC$ and $B$ of category $\mcD$,
  $\hom_{\mcC}(A,G-)$ is naturally isomorphic to $\hom_{\mcD}(FA,-)$ and $\hom_{\mcC}(-,GB)$ is naturally isomorphic to  $\hom_{\mcD}(F-,B)$.
  That is for any morphisms $f:C \to D$ of category $\mcC$ and $g: \widetilde{D} \to \widetilde{C}$ of category $\mcD$ there exist natural transformations $\eta$, $\mu$ such that the following diagrams
  \begin{center}
  \begin{tikzcd}
    \hom_{\mcC}(C, GB) \arrow[d, "{\hom_{\mcC}(-, GB)}"] \arrow[r, "{\eta_C}"]
    & \hom_{\mcD}(FC, B) \arrow[d, "{\hom_{\mcD}(F-, B)}"] \\
    \hom_{\mcC}(D, GB) \arrow[r, "{\eta_D}"]
    & \hom_{\mcD}(FD, B) 
  \end{tikzcd}
  \end{center}
  \begin{center}
\begin{tikzcd}
    \hom_{\mcC}(A, G\widetilde{D}) \arrow[d, "{\hom_{\mcC}(A, G-)}"] \arrow[r, "{\mu_{\widetilde{D}}}"]
    & \hom_{\mcD}(FA, \widetilde{D}) \arrow[d, "{\hom_{\mcD}(FA, -)}"] \\
    \hom_{\mcC}(A, G\widetilde{C}) \arrow[r, "{\mu_{\widetilde{C}}}"]
    & \hom_{\mcD}(FA, \widetilde{C}) 
  \end{tikzcd}
\end{center}
  commute, and each of the components of $\eta$ and $\mu$ are isomorphisms, i.e. $\eta_C$, $\eta_D$, $\mu_{\widetilde{D}}$, $\mu_{\widetilde{C}}$ are isomorphisms. In this case, we say $F$ is a left adjoint of $G$ and $G$ is a right adjoint of $F$.
\end{defn}
\begin{defn}
  \textbf{Coherence conditions} are a collection of conditions requiring that various compositions of elementary morphisms are equal. 
\end{defn}
\begin{defn}
  Let $\mcC$ be a category. A \textbf{monad }on $\mcC$ is given by an endofunctor $T: \mcC \to \mcC$ together with natural transformations $\eta: \mathbbm{1}_\mcC \to T$ (where $\mathbbm{1}_\mcC$ denotes the identity functor on $\mcC$ and
  $\mu: T^2 \to T$ (where $T^2$ is the functor $T \circ T$ from $\mcC$ to $\mcC$). These are required to fulfill the following coherence conditions:
  \begin{itemize}
  \item $\mu \circ T \mu = \mu \circ \mu T$ (as natural transformations $T^3 \to T$), i.e. for all objects $A$ of $\mcC$ the following diagram commutes,
    \begin{center}
      \begin{tikzcd}
        T^3 \arrow[d, "{\mu T}"] \arrow[r,"{T \mu}"] & T^2 \arrow[d, "{\mu}"] \\
        T^2 \arrow[r, "{\mu}"] & T
      \end{tikzcd}
    \end{center} 
    \begin{center}
      \begin{tikzcd}
        T(T(T(A))) \arrow[d, "{\mu_{T(A)}}"] \arrow[r,"{T(\mu_A)}"] & T(T(A)) \arrow[d, "{\mu_A}"] \\
        T(T(A)) \arrow[r, "{\mu_A}"] & T(A)
      \end{tikzcd}
    \end{center} 
  \item $\mu \circ T \eta = \mu \circ \eta T = \mathbbm{1}_T $ (as natural transformations  $T \to T$, where $\mathbbm{1}_T$ denotes the identity transformation from $T$ to $T$), i.e.
    the following diagram commutes, for all objects $A$ of $\mcC$:
    \begin{center}
      \begin{tikzcd}
        T \arrow[d, "{T \eta}"] \arrow[r,"{\eta T}"] & T^2 \arrow[d, "{\mu}"] \\
        T^2 \arrow[r, "{\mu}"] & T
      \end{tikzcd}
    \end{center} 
    \begin{center}
      \begin{tikzcd}
        T(A) \arrow[d, "{T(\eta_A)}"] \arrow[r,"{\eta_{T(A)}}"] & T(T(A)) \arrow[d, "{\mu_A}"] \\
        T(T(A)) \arrow[r, "{\mu_A}"] & T(A)
      \end{tikzcd}
    \end{center} 

  \end{itemize}
\end{defn}

\begin{defn}
  Let $\langle T, \eta, \mu \rangle$ be a monad over a category $\mcC$. The \textbf{Kleisli Category }of $\mcC$ is the category $\mcC_T$ whose objects and morphisms are given by:
  \begin{itemize}
  \item $\ob(\mcC_T) = \ob(\mcC)$,
  \item $\hom_{\mcC_T}(X, Y) = \hom_{\mcC}(X, TY)$.
  \end{itemize}
  where for $f: X \to TY$ and $g: Y \to TZ$, composition of morphisms is given by:
  \begin{center}
    $g \circ_T f = \mu_Z \circ Tg \circ f: X \to TY \to T^2Z \to TZ$
  \end{center}
  and the identity is satisfied by the monad unit $\eta$, i.e. $id_X = \eta_X$.
\end{defn}

\begin{defn}
  Let $\mcC$ be a category, $Z$ and $Y$ objects of $\mcC$, and let $\mcC$ have all binary products with $Y$. An object $Z^Y$ together with a morphism $eval: (Z^Y \times Y) \to Z$ is an exponential object if for any object $X$ and morphism $g: X \times Y \to Z$ there is a unique morphism $\lambda g : X \to Z^Y$ such that the following diagram commutes: 
  \begin{center}
    \begin{tikzcd}
      X \times Y \arrow[d, swap, "{\lambda g \times id_Y}"] \arrow[rd, "g"] \\
      Z^Y \times Y \arrow[r, swap, "eval"] & Z
    \end{tikzcd}
  \end{center} 
  In the category of sets, the exponential map $Z^Y$ is the set of all functions $Y \to Z$. The map $eval:(Z^Y \times X) \to Z$ is the evaluation map that sends $(f, y) \mapsto f(y)$. For any map $g: (X \times Y) \to Z$ the map
  $\lambda g: X \to Z^Y$ is the curried form of $g$:
  \begin{center}
    $\lambda g(x)(y) = g(x,y)$
  \end{center}
  In functional programming languages, the morphism eval is often called apply.
\end{defn}


\end{document}
