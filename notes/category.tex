\documentclass[12pt]{report}
\usepackage[utf8]{inputenc}
\usepackage{mathtools}
\usepackage{tikz-cd}
\newcommand{\mcC}{\mathcal{C}}
\newcommand{\mcD}{\mathcal{D}}
\newcommand{\mor}{\text{mor}}
\newtheorem{thm}{Theorem}[chapter]
\newtheorem{defn}[thm]{Definition} %arguments encased in square brackets define counters. Here thm is counted according to chapter, and defn is counted according to theorem.


\begin{document}
\chapter{Functors}
\begin{defn}
  Two functors $F: \mcC \to \mcD, G: \mcD \to \mcC$ are adjoint if for any object $A$ of category $\mcC$ and $B$ of category $\mcD$,
  $\hom_{\mcC}(A,G-)$ is naturally isomorphic to $\hom_{\mcD}(FA,-)$ and $\hom_{\mcC}(-,GB)$ is naturally isomorphic to  $\hom_{\mcD}(F-,B)$.
  That is for any morphisms $f:C \to D$ of category $\mcC$ and $g: \widetilde{D} \to \widetilde{C}$ of category $\mcD$ there exist natural transformations $\eta$, $\mu$ such that the following diagrams
  \begin{center}
  \begin{tikzcd}
    \hom_{\mcC}(C, GB) \arrow[d, "{\hom_{\mcC}(-, GB)}"] \arrow[r, "{\eta_C}"]
    & \hom_{\mcD}(FC, B) \arrow[d, "{\hom_{\mcC}(F-, B)}"] \\
    \hom_{\mcC}(D, GB) \arrow[r, "{\eta_D}"]
    & \hom_{\mcD}(FD, B) 
  \end{tikzcd}
  \end{center}
  \begin{center}
\begin{tikzcd}
    \hom_{\mcC}(A, G\widetilde{D}) \arrow[d, "{\hom_{\mcC}(A, G-)}"] \arrow[r, "{\mu_{\widetilde{D}}}"]
    & \hom_{\mcD}(FA, \widetilde{D}) \arrow[d, "{\hom_{\mcC}(FA, -)}"] \\
    \hom_{\mcC}(A, G\widetilde{C}) \arrow[r, "{\mu_{\widetilde{C}}}"]
    & \hom_{\mcD}(FA, \widetilde{C}) 
  \end{tikzcd}
\end{center}
  commute, and each of the components of $\eta$ and $\mu$ are isomorphisms, i.e. $\eta_C$, $\eta_D$, $\mu_{\widetilde{D}}$, $\mu_{\widetilde{C}}$ are isomorphisms.

\end{defn}

\end{document}
